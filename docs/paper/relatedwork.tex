\section{RELATED WORKS}



\subsection{\ac{SNNs}}

\ac{SNNs} \cite{maass1996lower,maass1997networks} are referred to as the third generation of neural networks because of their potential to supersede deep learning methods in field of computational neuroscience and biologically plausible \ac{ML}. \ac{SNNs} are also thought to be more practical for data-processing tasks in which the data has a temporal component since the neurons which comprise \ac{SNNs} naturally integrate their inputs over time. Recently, neural networks have become  increasingly prominent in machine learning and artificial intelligence research. However \ac{SNNs} have not been able to reach its full potential due to their inherent complexity and computational requirements. So researchers have developed spiking neural network frameworks addressed to this issues. \textbf{Nengo} \cite{bekolay2014nengo} is often used to simulate high-level functionality of brains or brain regions as supporting simulation at the level of spikes, firing rates, or high-level, abstract neural behavior. \textbf{NeuCube} \cite{kasabov2014neucube} supports rate coding-based spiking networks and attempt to map spatiotemporal input data into three-dimensional \acs{SNN} architecture. \textbf{CARLsim3} \cite{beyeler2015carlsim} and \textbf{NeMo} \cite{fidjeland2009nemo} allow simulation of large spiking networks built from \cite{izhikevich2003simple} with realistic synaptic dynamics as their fundamental building block. And both framework accelerate computation using \ac{GPU}. \textbf{BindsNet} \cite{hazan2018bindsnet} is the most recently developed \acs{SNN} framework on top of \textbf{PyTorch} deep learning library. This framework allows users to build, train, and evaluate \ac{SNNs}. The learning of connection weights is supported by various algorithms from the biological learning literature \cite{markram1997regulation}. And separate modules composed this framework provide functions such as loading of ML datasets, encoding of raw data into spike train network inputs, plotting of network state variables and outputs, and evaluation of \acs{SNN}. In this paper, to train, visualize, and evaluate our network, We utilize \textbf{BindsNet} to extract optical flow from event streams.

